
% Chapter 2 File

\chapter{Tasks}

	\section{Task 1: Data-driven approaches}
	\label{datadriv}

		In the first part of the Thesis, I will use data-driven approach to construct a forecast model for the Chlorophyll concentration. Data-driven only the available data to construct a model, without any physical modeling. I focused on space-time geostatistical models, as they are well established tool for similar studies [ref]. 

		\subsection{Subtask 1.1: Space-time covariance models}
		\label{datadriv:covmod}

	   		In the first part, I will construct a family of space-time covariance functions, using the method of [ref], and fit it to the data. I can then use Kriging to derive a linear state-space model for the data, and a Kalman filter can be used for filtering a prediction future values. This subtask has been completed, and an article has been submitted.

		\subsection{Subtask 1.2: Extreme event processes}
		\label{datadriv:extreme}

			The chlorophyll data has a lot of very high values corresponding to blooms. Therefore, the Gaussian process hypotheses made in classical geostatistics is problematic. In this section, I investigate if alternative processes can be used to better predict extremes values of the dataset. 

		\subsection{Subtask 1.3: Deriving a covariance model from NPZ dynamics}
		\label{datadriv:npzcov}

			In the two previous subtasks, we will use predefined families of covariance models. In this part, we want to derive the space-time covariance model from a deterministic model, following the method of [ref] and [ref]. The model we will use is the ecological Nutrient-Phytoplankton-Zooplankton model (NZP) [ref]. We will compare this model with the previous two approaches in the context of Kalman filter.

	\section{Task 2: Physical-statistical modeling}
	\label{physstat}

		The second part of the Thesis will mix physical models and statistical models using an approach called hierarchical modeling. The resulting models are also called Physical-statistical models. Specifically, in Task 2, I will use an NPZ model and time-series of remotely sensed chloropyll observation to estimate the nutrient flux in a region of the northern Red Sea.

		\subsection{Subtask 2.1: Expectation-Maximization with NPZ model}
		\label{physstat:EM}

			In this subtask, I will design a hierarchical model for the chlorophyll concentration that uses the NPZ model. The parameters of the model and the state are estimated using the Expectation-Maximization (EM) algorithm, following the approach of [ref]. This task is in progress. 
			

		\subsection{Subtask 2.2: Monte-Carlo sampling}
		\label{physstat:bayes}

			In subtask, I will use a Monte-Carlo sampling approach to estimate the state and the parameters. In order to do it, I will need to define a prior on the parameters. 

		\subsection{Subtask 2.3: Gaussian mixture model smoothing}
		\label{physstat:GM}

			In this last subtask, the EM algorithm from subtask 2.1 in modified in order to use a Gaussian Mixture Model smoothing, instead of the regular Ensemble Kalman Smoothing.  

	\section{Task 3: Data Assimilation of a complex deterministic ecological model}
	\label{moddriv}

		In this task, I will use a complex ecological model (ERSEM) coupled with a 3D global circulation model. The goal will be to assimilate remotely sensed chlorophyll data and forecast the chlorophyll concentration, use an ensemble filter. The performances of this setup will be evaluated and compared to the results of Task 1.