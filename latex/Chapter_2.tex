
% Chapter 2 File

\chapter{Tasks}

	\section{Task 1: Data-driven approaches}
	\label{datadriv}

		In the first part of the Thesis, I will use data-driven approach to construct a forecast model for the Chlorophyll concentration. Data-driven only the available data to construct a model, without any physical modeling. I focused on space-time geostatistical models, as they are well established tool for similar studies [ref]. 

		\subsection{Subtask 1.1: Space-time covariance models}
		\label{datadriv:covmod}

	   		In the first part, I will construct a family of space-time covariance functions, using the method of [ref], and fit it to the data. I can then use Kriging to derive a linear state-space model for the data, and a Kalman filter can be used for filtering a prediction future values. This subtask has been completed, and an article has been submitted.

		\subsection{Subtask 1.2: Extreme event processes}
		\label{datadriv:extreme}

			% Or collaboration with Ying Sun?

			% \begin{itemize}
			% 	\item Theory of extreme event processes
			% 	\item Outline future works
			% \end{itemize}

		\subsection{Subtask 1.3: Deriving a covariance model from NPZ dynamics}
		\label{datadriv:npzcov}

			% In section \ref{datadriv:covmod}, we chose the covariance model from a parametric family. However, in particular cases, in can be derived from a differential equation model, like in [ref] for the heat equation. The NPZ model (described in detail in chapter \ref{physstat}) can be coupled to a diffusion advection model to give a dynamical 2D process. In this part, I will derive the covariance model from this coupled model following the approach of [ref]. I will also compare its performance to the model fitted in \ref{datadriv:covmod} for prediction in a Kalman filter. 

	\section{Task 2: Physical-statistical modeling}
	\label{physstat}

		\subsection{Subtask 2.1: Expectation-Maximization with NPZ model}
		\label{physstat:EM}

			<paper with Boujeema>

		\subsection{Subtask 2.2: Bayesian approach}
		\label{physstat:bayes}

			\begin{itemize}
				\item Motivate Bayesian approach
				\item Outline future works
			\end{itemize}

		\subsection{Subtask 2.3: Gaussian mixture model smoothing}
		\label{physstat:GM}

			\begin{itemize}
				\item Motivate Gaussian mixture model
				\item Outline future works
			\end{itemize}

	\section{Task 3: Data Assimilation of a complex deterministic ecological model}
	\label{moddriv}

		\begin{itemize}
			\item Introduction to ERSEM model
			\item Summarize preceding works
			\item Outline future works
		\end{itemize}