
% Chapter 2 File

\chapter{Tasks}

	\section{Task 1: Data-driven approaches}
	\label{datadriv}
		
		% Data-driven approaches are modelling approaches that only make use of data, without any assumption on underlying physical phenomenons. This type of methods is particularly useful for modelling data generated by unknown of very complex physical processes. It include a wide variety of approaches coming from the fields of statistics and stochastic processes. 

		% The chlorophyll concentration is the result of the coupling of many complex ecological, biological, chemical and physical dynamics. Therefore, it seems appropriate to start modeling with data-driven approaches.

		% I start by presenting the use of a space-time covariance models to derive a linear model, trained on remotely-sensed data, that is used with a Kalman filter for forecasting (section \ref{datadriv:covmod}. This part has been submitted for publication. The following two sections present future extensions on this work: section \ref{datadriv:extreme} will present a modification of the scheme to better predict the blooms; in section \ref{datadriv:npzcov}, I will derive the space-time covariance model from a simple ecological model.

		\subsection{Subtask 1.1: Space-time covariance models}
		\label{datadriv:covmod}

	    % < summarize paper here >
	    % Should this part be close to the paper, or does it have integrated with the rest of the chapter?

		\subsection{Subtask 1.2: Extreme event processes}
		\label{datadriv:extreme}

			% Or collaboration with Ying Sun?

			% \begin{itemize}
			% 	\item Theory of extreme event processes
			% 	\item Outline future works
			% \end{itemize}

		\subsection{Subtask 1.3: Deriving a covariance model from NPZ dynamics}
		\label{datadriv:npzcov}

			% In section \ref{datadriv:covmod}, we chose the covariance model from a parametric family. However, in particular cases, in can be derived from a differential equation model, like in [ref] for the heat equation. The NPZ model (described in detail in chapter \ref{physstat}) can be coupled to a diffusion advection model to give a dynamical 2D process. In this part, I will derive the covariance model from this coupled model following the approach of [ref]. I will also compare its performance to the model fitted in \ref{datadriv:covmod} for prediction in a Kalman filter. 

	\section{Task 2: Physical-statistical modeling}
	\label{physstat}

		\subsection{Subtask 2.1: Expectation-Maximization with NPZ model}
		\label{physstat:EM}

			<paper with Boujeema>

		\subsection{Subtask 2.2: Bayesian approach}
		\label{physstat:bayes}

			\begin{itemize}
				\item Motivate Bayesian approach
				\item Outline future works
			\end{itemize}

		\subsection{Subtask 2.3: Gaussian mixture model smoothing}
		\label{physstat:GM}

			\begin{itemize}
				\item Motivate Gaussian mixture model
				\item Outline future works
			\end{itemize}

	\section{Task 3: Data Assimilation of a complex deterministic ecological model}
	\label{moddriv}

		\begin{itemize}
			\item Introduction to ERSEM model
			\item Summarize preceding works
			\item Outline future works
		\end{itemize}