

\chapter{Introduction}
\label{intro}

% TODO
% - Introduce 'CHL'?

% Introducing the subject: Ecosystem modeling
The study of ecosystem has increasingly relied on computer models during the last decades [ref]. They have become indispensable for designing observation networks [ref], interpreting data [ref] or estimating quantities of interest [ref]. Ecosystems can be, however, extremely complex, and selecting the right simplifications is a complex for the modeler, and very much depends on the question that need to be answered. 

% Introducting the subject: Chl modeling
My PhD work focused on the particularly challenging modeling of marine ecosystems. I worked on the modeling of phytoplankton, which plays a crucial role, as it is at the base of the marine food chain. Working with remote-sensing data of the Red Sea, I was trying to understand what affects the phytoplankton concentration, and to which extent we can predict its population in the future. 

% Outlining the introduction
Section \ref{intro:context} provides details about phytoplankton: its biology and importance (section \ref{intro:context:generalities}), the measurements of chl (section \ref{intro:context:sensing}), particularities of the CHL in the Red Sea (section \ref{intro:context:chlredsea}), and the different approaches existing for phytoplankton modeling (section \ref{intro:context:modeling}). Section \ref{intro:context} summarizes the main open questions that this thesis is adressing. Section \ref{intro:objectives} further details the expected contribution of this work. Finally, section \ref{intro:outline} introduces the different parts of the thesis. 

\section{Context}
\label{intro:context}

	\subsection{Generalities about phytoplankton}
	\label{intro:context:generalities}

		% Facts about phytoplanktons
		Phytoplankton are small, unicellular, photosynthetic algae. Thus they depend on the sunlight and mineral nutrients to survive. The three main types of phytoplankton are diatoms, coppecod, and cyanobacterias. Diatoms typically mesure between xxx and xxx, they are usually predominant in nutrient rich environments. Coppecod mesure between xxx and xxx, and are characterized by there motility. Finally, cyanobacterias are very small (xxx to xxx), and are able to survive in environments with very little nutrient, like the Red Sea. 

		% Role in Marine food chain
		Phytoplakton are primary producers: they produce biomass from their mineral nutrients. They are at the base of the marine food chain, and most of the marine life depends on them. They are therefore of crucial important to understand the marine ecology. 

		% Importance for fisheries
		Importance for fisheries

		% Role for climate
		Phytoplankton plays an important role for the earth climate. They trap CO2 during photosynthesis, and, when they die, they sink to the bottom of the ocean to form sediments. They therefore play the role of a biological pump. It is believed that their importance for regulating the climate is comparable to the land forests [ref].

	\subsection{Remote sensing of CHL}
	\label{intro:context:sensing}

		\begin{itemize}
	      	\item History of CHL remote sensing
	      	\item Principles of CHL remote sensing
	      	\item Processing of the data
	      	\item Limitations
	      	\item Other way of measuring phytoplankton			
		\end{itemize}

	\subsection{CHL in the Red Sea}
	\label{intro:context:chlredsea}

		\begin{itemize}
			\item Introducing the subsection
		\end{itemize}

		\subsubsection{The Red Sea}

			\begin{itemize}
				\item Red Sea physics
				\item Red Sea ecology
				\item Red Sea exploration
				\item Ecosystem threats
			\end{itemize}

		\subsubsection{Phytoplankton in tropical and subtropical environment}

			\begin{itemize}
				\item Oligotrophy
				\item Coral reefs
			\end{itemize}

		\subsubsection{Phytoplankton in the Red Sea}

			\begin{itemize}
				\item Data availability
				\item Sources of nutrient
				\item Spatial patterns
				\item Temporal patterns
				\item Extreme blooms
			\end{itemize}

	\subsection{Modeling of CHL}
	\label{intro:context:modeling}

		\begin{itemize}
			\item Introducing the two kinds of approaches
		\end{itemize}

		\subsubsection{Deterministic approaches}

			\begin{itemize}
				\item NPZ/D
				\item ERSEM
			\end{itemize}

		\subsubsection{Statistical approaches}

			\begin{itemize}
				\item Neural networks
				\item EOF
				\item Time series
			\end{itemize}

\section{Open Questions}
\label{intro:questions}

	\begin{itemize}
		\item Sources of nutrients
		\item Seasonal blooms
		\item Extreme Blooms
		\item Predictability of CHL
	\end{itemize}

\section{Objectives and contributions}
\label{intro:objectives}

	\begin{itemize}
		\item Predictability of CHL in Red Sea
		\item Sources of nutrients
		\item Causes of extreme blooms
		\item Best approache for CHL forecasting in Red Sea
	\end{itemize}

\section{Outline}
\label{intro:outline}

	\begin{itemize}
		\item data-driven approach
		\item mixed approach
		\item model-driven approach
	\end{itemize}
