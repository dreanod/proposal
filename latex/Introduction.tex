

\chapter{Introduction}
\label{intro}

% TODO
% - Introduce 'CHL'?

% Introducing the subject: Ecosystem modeling
The study of ecosystem has increasingly relied on computer models during the last decades [ref]. They have become indispensable for designing observation networks [ref], interpreting data [ref] or estimating quantities of interest [ref]. Ecosystems can be, however, extremely complex, and selecting the right simplifications is a complex for the modeler, and very much depends on the question that need to be answered. 

% Introducting the subject: Chl modeling
My PhD work focused on the particularly challenging modeling of marine ecosystems. I worked on the modeling of phytoplankton, which plays a crucial role, as it is at the base of the marine food chain. Working with remote-sensing data of the Red Sea, I was trying to understand what affects the phytoplankton concentration, and to which extent we can predict its population in the future. 

% Outlining the introduction
Section \ref{intro:context} provides details about phytoplankton: its biology and importance (section \ref{intro:context:generalities}), the measurements of chl (section \ref{intro:context:sensing}), particularities of the CHL in the Red Sea (section \ref{intro:context:chlredsea}), and the different approaches existing for phytoplankton modeling (section \ref{intro:context:modeling}). Section \ref{intro:context} summarizes the main open questions that this thesis is adressing. Section \ref{intro:objectives} further details the expected contribution of this work. Finally, section \ref{intro:outline} introduces the different parts of the thesis. 

\section{Context}
\label{intro:context}

	\subsection{Generalities about phytoplankton}
	\label{intro:context:generalities}

		% Facts about phytoplanktons
		Phytoplankton are small, unicellular, photosynthetic algae. Thus they depend on the sunlight and mineral nutrients to survive. The three main types of phytoplankton are diatoms, coppecod, and cyanobacterias. Diatoms typically mesure between xxx and xxx, they are usually predominant in nutrient rich environments. Coppecod mesure between xxx and xxx, and are characterized by there motility. Finally, cyanobacterias are very small (xxx to xxx), and are able to survive in environments with very little nutrient, like the Red Sea. 

		% Role in Marine food chain
		Phytoplakton are primary producers: they produce biomass from their mineral nutrients. They are at the base of the marine food chain, and most of the marine life depends on them. They are therefore of crucial important to understand the marine ecology. 

		% Importance for fisheries
		Importance for fisheries

		% Role for climate
		Phytoplankton plays a crucial role for the earth climate. They trap CO2 during photosynthesis, and, when they die, they sink to the bottom of the ocean to form sediments. They therefore play the role of a biological pump. It is believed that their importance for regulating the climate is comparable to the land forests [ref].

	\subsection{Remote sensing of CHL}
	\label{intro:context:sensing}
		% In situ measurements
		Principles of in-situ measurements. Main types. Advantages: Column resolution, temporal resolution. Limitation: spatial resolution: impossible to see large scale phenomenons. 

		% History of CHL remote sensing
		When did chl remote sensing started. Why people got this idea. First missions and its results. Seawifs. Modis. New missions. 

		% Principles of CHL remote sensing
		Optical principles. Wavelengths. Max Depth. 

		% Processing of the data
		Different algorithms to process the data. 

		% Limitations
		Known limitations of chl remote sensing. Maximum CHL not visible. Low bathymetries. 

	\subsection{CHL in the Red Sea}
	\label{intro:context:chlredsea}

		As my PhD focused on the Red Sea, I will introduce here some notions about this peculiar environment and patterns of variation for the CHL in the Red Sea. 

		\subsubsection{The Red Sea}

			% The Red Sea physics
			The Red Sea is a remarkable marginal sea situated between Africa and the Arabic Peninsula and connected to the Mediterranean Sea through the Suez canal, and to the Gulf of Aden through the strait of Bab-el-Mandeb. It is one the most saline bodies of water in the world [ref] with a salinity of xxx. It also one of the warmest seas in the world, with surface temperatures between ... and ... [ref]. The average depth of the Red Sea is ..., but reaches ... in the trench that ranges it from north to south. This trench is the place where the Arabian and the African plates are drifing away [ref]. The southern Red sea is shallower (mean depht of ...) than the nothern red Sea (mean depth ... ).

			% Red Sea ecology
			The Red Sea is the northernmost tropical sea and is home to a unique but fragile ecosystem. xxxx different species of fishes where observed in its waters, from which ... species are endemic. Most of this biodiversity can be found in its .... km  of coral reefs [ref]. This unique resource is however threatean by human activities as the touristic exploitation and the economy develop along its shore [ref]. The impacts of the climate change on the coral reefs is also a subject of worries [ref]. 

			% Red Sea exploration
			To this date, the Red Sea has been relatively little studied. Political and economical difficulties have made it difficulties have prevented the deployement of long term and global monotoring mission. Research on this area therefore relies highly on computer simulations and remotely sensed data.

		\subsubsection{Phytoplankton in tropical and subtropical environment}

			Tropical and subtropical seas are generally poor in nutrients (oligotrophic). Because the water is much warmer at the surface than below, the water column is highly stratified. This means that the surface water hardly mixes with the water below it. Since the phytoplankton needs to be near the surface to photosynthesize, primary production (the production of phytoplankton from new nutrients) is limited by the slow upward diffusion of nutrients [ref].

			The rich ecodiversity of coral reefs contrasts surpringly with the low productivity of neighboring waters. The capacity of coral reefs to sustain a high primary productivity in spite of oligotrophic waters has been noticed by Darwin [ref] and is coined "Darwin's paradox". It is now believed that this ecological success is due to the many symbiotic relationships existing in a coral reef that form a efficient recyling system for nutrients [ref]. 

		\subsubsection{Phytoplankton in the Red Sea}

			% Sources of nutrients
			The Red Sea is an extremely oligotrophic environment. It is surrounded by arid lands and has no significant river outlets. The only significant source of fresh water is the Gulf of Aden through the strait of Bab-el-Mandeb in the southern extremity. Dust brought by winds is also believed to carry nutrients [ref]. Subsurface waters are also thought to diffuse nutrients to the surface waters during winter episodes of deep mixing, especially in the north [ref].

			% Spatial patterns
			The annual average of chlorophyll concentrations in the Red Sea shows a north-south gradient, with higher concentrations in the south [ref]. The lowest concentrations are found between latitudes 22$^\circ$N and 26$^\circ$N. Some sparse areas of high concentrations can be found near the coasts, where the coral reefs are [ref]. Overall this repartition is stable in time. 

			% Temporal patterns
			Chlorophyll concentrations in the Red Sea display a strong seasonal behaviour. The concentrations reach their minima during the summer and their maxima during the winter bloom. This peak happens in xxx in the southern Red Sea, and in xxx in the northern Red Sea. There is however an important internannual variability making it diffult to identify long-term trends. 

			% Extreme blooms
			Despite its oligotrophic waters, the chlorophyll concentrations can become locally several orders of magnitude higher than the seasonal average. These episodes of extreme blooms (to be distinguished from the seasonal winter bloom) have a lifetime ranging from a few days to a few weeks [ref]. The reasons behind these sudden blooms is not understood yet. 

	\subsection{Modeling of CHL}
	\label{intro:context:modeling}

		The modeling of phytoplankton and chlorophyll observations has used large diversity of approach. Deterministic approaches, inspired by a physical approach to ecological modeling has given birth to a large quantity of models, from very simple to extremely complex. Statistical and machine learning approaches have also been used for studies focusing on very specific questions. 

		\subsubsection{Deterministic approaches}

			% NPZ/D
			One of the simplest model for phytoplankton growth is the Nutrient-Phytoplankton-Zooplankton (NPZ) model. It is uses Ordinary Differential Equations (ODE) to simulate the relative quantity of each of the state variables $N$ (nutrient), $P$ (phytoplankton) and $Z$ (zooplankton) in terms of a chemical specie. I provide details on the NPZ model in section \ref{physstat:EM}. There is an infinity of variants to the NPZ model. One can use the NPD model, where D stands for "Detritus". Different groups of phytoplankton can be differentiated, and one can have $P_1$, $P_2$ and $P_3$ for diatoms, flagellates and cyanobacteria. Differents species between these groups can be represented as well as higher trophic levels, etc.

			% ERSEM
			On this opposite side of the NPZ model in this range of refinements, one can find very sophisticated models such as ERSEM (European Regional Seas Ecological Model). Such models allow for a complete ecosystem simulation on the scale of a regional sea, such a the North Sea [ref] or the Red Sea [ref]. They are however very difficult to calibrate and operate, due to the enormous number of variables needed [ref]. There is therefore a debate on their usefulness compared to smaller models [ref]. 

		\subsubsection{Statistical approaches}

			% Time series
			Time series records of biological and physical measurements such as Bermuda Atlantic Time-Series (BATS) have been made publicly available since the 50s, and have inspired the use of time series statistical tools for primary production. Well known examples are [ref]. Forecasting the chlorophyll concentrations have used ... [ref].

			% EOF
			Empirical Orthogonal Functions (EOF) are a useful tool for the exploration of large space-time datasets. Their use for remotely-sensed chlorophyll data has been made possible thanks to data filling algorithms like DINEOF. DINEOF is explained in more details in section \ref{datadriv:covmod}. EOF analysis for chlorophyll concentration have been made in the Gulf of Alaska [ref], the United States South Atlantic Bight[ref] and the Mediterranean [ref].

			% Neural networks
			Neural networks, a machine learning algorithm, has been used to predict phytoplankton blooms from remotely-sensed data. Examples studies include [ref]

\section{Open Questions}
\label{intro:questions}

	In the previous section, I summarized the state of the knowledge for large scales dynamics of phytoplankton concentration in general, in the Red Sea in particular. In this section, I would like to emphasize some open questions. 

	Despite the Red Sea being extremely oligotrophic, very important bloom episode can take place in it. The causes of these extreme blooms has not been found. Several hypotheses have been made: dust storms, wind-induced mixing, eddies, etc. To this date however, the source of nutrients for these blooms has not been found.

	The presence of these unexplained blooms causes a problem for predicting the chlorophyll and phytoplankton concentrations. Currently, simulations in the Red Sea are able to resolve the seasonal dynamics. However being, able to forecast at shorter terms is desirable in order to better understand the Red Sea ecosystem. 

\section{Objectives}
\label{intro:objectives}
	
	The first objective of my work is to identify possible causes of extreme blooms in the Red Sea. In particular, I will try to see if extreme blooms and seasonal blooms are driven by the same factors. To answer this question, it will be necessary to develop precise definitions for extreme blooms and seasonal blooms. 

	The second objective is to develop approaches to forecast chlorophyll levels in the Red Sea. Different ranges for chlorophyll forecasted will have to be chosen. The different solutions for forecasting also need to be compared to a common measure. 

\section{Outline}
\label{intro:outline}

	There are two major paradigms that one can use to model space-time datasets: data-driven or model-driven. Data-driven approaches include statistical or machine learning modeling techniques. They are generic methods that can be applied to widely different a unrelated classes of problem, because they make no assumption on the phenomenon behind the data. Model-driven approaches, on the contrary, simulate the physical processes that generate the data. Finally, the recent advent of hierarchical modeling has allowed the development of mixed approaches.

	% In chapter \ref{datadriv}, I explore data-driven approaches. I summarize a study I have done on space-time covariance models used in conjunction with a Kalman filter for forecasting chlorophyll concentration anomalies. This work has been submitted for publication. Future extensions to this work will be outlined.

	% In chapter \ref{physstat}, I outline a study currently ongoing on the use of a simple NPZ model embedded in a hierarchical model for the northern Red Sea ecosystem state a parameters estimation. The first part of this study, that uses an Expectation-Maximization estimation with a smoothing ensemble algorithm, is in preparation for publication. Future works will include the use of a sequential Monte-Carlo sampler and a Gaussian Mixture model.

	% In chapter \ref{moddriv}, a future work is outlined, that will use a complex coupled ecosystem model to forecast the chlorophyll concentration in the Red Sea. This will allow us to identify the factors explaining the extreme blooms in the regions.
